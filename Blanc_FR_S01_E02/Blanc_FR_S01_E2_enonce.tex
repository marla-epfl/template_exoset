%--------------------------------------------------------------------------
%
%                                                    ENONCE
%
%--------------------------------------------------------------------------

\noindent {\Large {\bf Saut du saumon}}\\[-4mm]
\begin{enumerate}
\item Montrer que la solution générale de l'équation du mouvement rectiligne uniform\'ement acc\'el\'er\'e, le long d'un axe $x$,  
\[
\ddot{x} = a ,
\]
où $a$ est une constante, est donn\'ee par $x(t) = \frac{1}{2}at^2 + v_0t + x_0$ quelles que soient les constantes $v_0$ et $x_0$. Interpr\'eter ces constantes. 

\item Un saumon saute hors d'un lac avec une vitesse initiale $v_0$ dirig\'ee verticalement vers le haut. Il subit une acc\'el\'eration constante \'egale \`a $-g$, due \`a la pesanteur. Repr\'esenter graphiquement la position verticale du poisson en fonction du temps, ainsi que sa vitesse en fonction du temps.

\item Quelle hauteur maximale le saumon atteindra-t-il ? Combien de temps passera-t-il en l'air ?

\emph{Application num\'erique:} $v_0 = 3\, {\rm m/s}$ et $g = 10\, {\rm m/s}^2$.  \\

\end{enumerate}