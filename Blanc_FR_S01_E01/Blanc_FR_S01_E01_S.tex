%--------------------------------------------------------------------------
%
%                                                    SOLUTION
%
%--------------------------------------------------------------------------


\noindent {\large {\bf Questions conceptuelles}}\\[-3mm]
\begin{enumerate}
\item La figure nous indique que les positions successives des deux blocs, à des intervalles de temps réguliers, sont séparées par une même distance. On en déduit que la vitesse de chaque bloc est constante et donc que leur accélération est nulle.

\item Soit $x$ un axe parallèle à PQ et $y$ un axe perpendiculaire à PQ, dans la direction de l'accélération durant la deuxième phase du mouvement. La composante $x$ de la vitesse du vaisseau, $v_x$, doit rester constate, car il n'y a jamais d'accélération selon $x$. Les trajectoires 1 et 2, qui correspondent à une vitesse $v_x$ qui s'annule tout à coup au point Q, ne sont donc pas possibles. Après le point Q, la trajectoire doit correspondre à celle d'un mouvement uniformément accéléré, c'est-à-dire à une parabole. Les trajectoires 2, 3, et 4 ne sont manifestement pas paraboliques. La seule possiblité est donc la trajetoire 5, qui est parabolique dàs le point Q et compatible avec une vitesse $v_x$ constante.

\end{enumerate}
